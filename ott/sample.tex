\documentclass{article}

\usepackage{xcolor, url}
\usepackage[fleqn]{amsmath}

%% == OTT =======================================================
\usepackage[supertabular,lineBreakHack]{ottalt}
\inputott{oolong.ott.tex}

%% == Syntax Macros==============================================
\newcommand{\npe}{\textbf{NullPointerException}}
\newcommand{\async}[3]{#1 \mathop{||} #2 \rhd #3}
\newcommand{\thread}[2]{(#1, #2)}
\newcommand{\locks}{\ensuremath{\mathcal{L}}}
\newcommand{\exn}{\textbf{EXN}}
\newcommand{\cfg}[3]{\langle #1; #2; #3\rangle}
\newcommand{\Id}{\ensuremath{\mathit{Id}}}
\newcommand{\Ids}{\ensuremath{\mathit{Ids}}}
\newcommand{\Cd}{\ensuremath{\mathit{Cd}}}
\newcommand{\Cds}{\ensuremath{\mathit{Cds}}}
\newcommand{\Msig}{\ensuremath{\mathit{Msig}}}
\newcommand{\Msigs}{\ensuremath{\mathit{Msigs}}}
\newcommand{\Fd}{\ensuremath{\mathit{Fd}}}
\newcommand{\Md}{\ensuremath{\mathit{Md}}}
\newcommand{\Fds}{\ensuremath{\mathit{Fds}}}
\newcommand{\Mds}{\ensuremath{\mathit{Mds}}}
\newcommand{\SB}{\startblock}
\newcommand{\FB}{\finishblock}
\newcommand{\Cfg}{\ensuremath{\mathit{cfg}}}
\renewcommand{\SB}{\startblock}
\renewcommand{\FB}{\finishblock}

\renewcommand{\c}[1]{\texttt{#1}}
\renewcommand{\kw}[1]{\textbf{#1}}


\title{Sample Document Using OOlong}

\begin{document}

\maketitle

\section{Examples}

This document exemplifies how to extract the \LaTeX{} version of
the OOlong semantics from Ott and our hand-written grammars.
Figure \ref{fig:syntax} shows the syntax of OOlong. The
corresponding \LaTeX{} code is in \texttt{syntax.tex}. Figure
\ref{fig:runtime_syntax} shows the syntax of OOlong. The
corresponding \LaTeX{} code is in \texttt{runtimeSyntax.tex}.

Figure \ref{fig:wf_program} shows the well-formedness rules for
classes and interfaces. The type rules are generated using the
\c{\textbackslash drules} command, defined by Ott. It is also
possible to include single rules without the headers in the
figure by using the \c{\textbackslash drule} command:\medskip

\drule{wfXXlet}\medskip

The \LaTeX{} sources for the type rules are generated from the
file \c{oolong.ott} (by running the makefile in this directory).
They look extra nice because we are using the \c{ottalt}
package\cite{ottalt}! In order to use the \c{\textbackslash drule}
and \c{\textbackslash drules} commands, the preamble must contain
\c{\textbackslash inputott\{oolong.ott.tex\}}.

\begin{figure}
  \centering
  \begin{tabular}{lcl}
  $P$ & $::=$ & \Ids~\Cds~$e$ \hfill \textit{(Programs)}\\
  \Id & $::=$ & \kw{interface} $I$ \SB \Msigs \FB \hfill \textit{(Interfaces)} \\
      & $|$ & \kw{interface} $I$ \kw{extends} $I_1, I_2$ \\
  \Cd & $::=$ & \kw{class} $C$ \kw{implements} $I$ \SB \Fds~\Mds \FB \hfill \textit{(Classes)}\\
  \Msig & $::=$ & $m(x : t_1) : t_2$ \hfill \textit{(Signatures)} \\
  \Fd & $::=$ & $f : t$ \hfill \textit{(Fields)} \\
  \Md & $::=$ & \kw{def} \Msig \SB$e$\FB \hfill \textit{(Methods)} \\
  $e$ & $::=$ & $v$
        $~|~$ $x$
        $~|~$ $x.f$
        $~|~$ $x.f$ \c{=} $e$
        $~|~$ $x.m(e)$ \hfill \textit{(Expressions)} \\
      & $~|~$ & \kw{let} $x$ \c{=} $e_1$ \kw{in} $e_2$
        $~|~$ \kw{new} $C$
        $~|~$ $(t)~e$ \\
      & $~|~$ & \kw{finish}\SB \kw{async}\SB$e_1$\FB~\kw{async}\SB$e_2$\FB \FB\c{;} $e_3$ \\
      & $~|~$ & \kw{lock}$(x)$ \kw{in} $e$
        $~|~$ \colorbox{lightgray}{\kw{locked}$_\iota$\SB$e$\FB}\\
  $v$ & $::=$ & \kw{null}
        $~|~$ \colorbox{lightgray}{$\iota$} \hfill \textit{(Values)} \\
  $t$ & $::=$ & $C$
        $~|~$ $I$
        $~|~$ $\mathbf{Unit}$ \hfill \textit{(Types)} \\
  $\Gamma$ & $::=$ & $\epsilon$
             $~|~$ $\Gamma, x : t$
             $~|~$ \colorbox{lightgray}{$\Gamma, \iota : C$} \hfill \textit{(Typing environment)}
\end{tabular}

  \caption{Syntax of OOlong }
  \label{fig:syntax}
\end{figure}

\begin{figure}
  \centering
  \begin{tabular}{lcl}
  \Cfg{} & $::=$ & $\cfg{H}{V}{T}$ \hfill \textit{(Configuration)} \\
     $H$ & $::=$ & $\epsilon$
           $~|~$ $H, \iota \mapsto obj$ \hfill \textit{(Heap)} \\
     $V$ & $::=$ & $\epsilon$
           $~|~$ $V, x \mapsto v$ \hfill \textit{(Variable map)} \\
     $T$ & $::=$ & $(\locks, e)$
           $~|~$ $\async{T_1}{T_2}{e}$
           $~|~$ \exn \hfill ~~\textit{(Threads)}\\
   $obj$ & $::=$ & $(C, F, L)$ \hfill \textit{(Objects)}\\
     $F$ & $::=$ & $\epsilon$
           $~|~$ $F, f \mapsto v$ \hfill \textit{(Field map)} \\
     $L$ & $::=$ & $\mathbf{locked}$
           $~|~$ $\mathbf{unlocked}$ \hfill \textit{(Lock status)}\\
  $\exn$ & $::=$ & $\npe$ \hfill \textit{(Exceptions)}\\
\end{tabular}

  \caption{Syntax of runtime constructs of OOlong}
  \label{fig:runtime_syntax}
\end{figure}

\begin{figure}
  \drules{$\vdash P:t\quad\vdash\mathit{Id}\quad\vdash\mathit{Cd}\quad\vdash\mathit{Fd}\quad\vdash\mathit{Md}$}
         {Well-formed program}
         {wfXXprogram
         ,wfXXinterface
         ,wfXXinterfaceXXextends
         ,wfXXclass
         ,wfXXfield
         ,wfXXmethod
       }
\caption{Well-formedness of classes and interfaces. }
  \label{fig:wf_program}
\end{figure}

\begin{figure}
  \[
\mathbf{vardom}(\Gamma) = \{x~|~x\in\mathbf{dom}(\Gamma)\}
\]
%
\[
\mathbf{msigs}(I) =
\left\{
\begin{array}{l l}
Msigs & \mbox{if } \kw{interface}~I\SB \mathit{Msigs} \FB \in P\\
\mathbf{msigs}(I_1)\cup\mathbf{msigs}(I_2) & \mbox{if } \kw{interface}~I~\kw{extends}~I_1,I_2 \in P\\
\end{array}
\right.
\]
%
\[
\mathbf{msigs}(C) = \{\mathit{Msig}~|~\kw{def}~\mathit{Msig}\SB e \FB \in Mds\}
\mbox{ if } \kw{class}~C \dots\SB \_~Mds \FB \in P
\]
%
\[
\mathbf{msigs}(t)(m) = x : t_1 \rightarrow t_2$ \mbox{ if } $m(x : t_1) : t_2 \in \mathbf{msigs(t)}
\]
%
\[
\mathbf{heldLocks}(T) =
\left\{
\begin{array}{l l}
\mathcal{L} & \mbox{if } T = (\mathcal{L}, e)\\
\mathbf{heldLocks}(T_1)\cup\mathbf{heldLocks}(T_2) & \mbox{if } T = \async{T_1}{T_2}{e}
\end{array}
\right.
\]
%
\[
\mathbf{locks}(e) = \{\iota~|~\kw{locked}_\iota\SB e' \FB \in e\}
\]
%
\[
\begin{array}{l}
\mathit{distinctLocks}(e) \equiv |\mathbf{locks}(e)| = |\mathbf{lockList}(e)|\\
\quad\mbox{where } \mathbf{lockList}(e) = [\iota~|~\kw{locked}_\iota\SB e' \FB \in e]
\end{array}
\]

  \caption{Helper functions}
  \label{fig:helpers}
\end{figure}

\begin{thebibliography}{}

\bibitem{ottalt}
  Jesse Tov,
  The \c{ottalt} \LaTeX{} package:\\
  \url{http://users.eecs.northwestern.edu/~jesse/code/latex/ottalt/ottalt.pdf}.

\end{thebibliography}

\end{document}